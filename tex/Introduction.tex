% !TEX root = MasterThesis.tex

\chapter{Introduction}\label{ch:introduction}

% What is Fuchsia and why is it of interest?

% Research Objectives:
%     not included:
%         Fuchsia itself, and everything UI related, 
%         Security as an own chapter (issued as part of the other topics), 
%         filesystems
%     included: 
%         Difference between monolithic kernel and micro kernel, 
        %   effects of a micro kernel on driver development 
        %   (includes tooling, workflow, ...)

Modern application \acp{cpu}, as used for e.g.\ desktop computers or smartphones, are highly complex ciruits consisting of at least two independent actual processing units.
It is hardly possible to master these multicore processors, associated memory and \ac{io} devices in their full complexity nor take advantage of their actual computing power or security mechanisms without using an operating system as an abstraction layer between the actual hardware and its user.
As such an operating system, to manage a computer's physical resources and share them between an amount of user programs in a fair and safe way, is the core task.
The part of those providing this is commonly referred as the \textit{kernel}.  
Even if this distinction is widely uses, especially for Linux, there is hardly no clear and global definition which tasks of an operating system are part of the \textit{kernel}.
\textsc{Albrecht Achilles} named process management, memory management and basic \ac{io} operations as such ones in his book\cite{achilles2006betriebssysteme}.
Actually, this distinction between a whole operating system and its kernel itself is that hard because it is not even consistent between different operating system architecture concepts.
The lot of different goals and use-cases for operating systems which have a big impact to the architectual design, and with this to the term \textit{kernel} too. 
But for a majority of them, at least core drivers are considered as part of the kernel. 

With \textit{Linux} and \textit{Fuchsia}, two fundamentally different represenatives of the two best-known operating system concepts, the monolith and the microkernels, will be considered in this work.
\textit{Fuchsia}, respectively the \textit{Zircon} kernel is a new microkernel operating system which is developed from scratch by \textit{Google}.
The project is developed in public since 2016\footcite{\url{https://www.androidpolice.com/2016/08/12/google-developing-new-fuchsia-os-also-likes-making-new-words/}}, but since then there has been no official announcement about the purpose of the system.
Nevertheless, there are lots of reasons to study \textit{Fuchsia} and especially to its kernel \textit{Zircon}.
The first one is clearly an architectual one, as \textit{Zircon} is a microkernel.
Of course are there some industrial used microkernel architectures, but the most and most successful ones are \ac{posix} compatible and targeting real time tasks, such as \textit{QNX} or \textit{FreeRTOS}. % TODO cite?
But apart from the unknown use-case, \textit{Zircon} differs in a very important aspect, the \ac{posix} compatibility which provides a global standard (DIN 9945, IEEE1003.2) for UNIX-like operating systems\cite{wolf2009c}.
That is a very brave decision as UNIX compatibility is a common part of public tenders for software.
One further point making \textit{Zircon} an interesting research topic is \textit{C++} as preferred programming language for the kernel, which is a rather unsual choice\cite{tanenbaum-modern-operating-systems}.

\textit{Linux} is in many issues very opposite to \textit{Zircon}.
It was started as a hobby project by Linus Torvalds in 1991 and growed over the years to a widely spread, powerful kernel.
Today, due to its use in \textit{Android}, is the most used operating system kernel at all. % TODO https://netmarketshare.com/operating-system-market-share.aspx?options=%7B%22filter%22%3A%7B%7D%2C%22dateLabel%22%3A%22Trend%22%2C%22attributes%22%3A%22share%22%2C%22group%22%3A%22platform%22%2C%22sort%22%3A%7B%22share%22%3A-1%7D%2C%22id%22%3A%22platformsDesktop%22%2C%22dateInterval%22%3A%22Monthly%22%2C%22dateStart%22%3A%222018-02%22%2C%22dateEnd%22%3A%222019-01%22%2C%22segments%22%3A%22-1000%22%7D
%TODO weiter
%clear definition between kernel and os
%Further kernel parts (driver, syscalls, io, ..., filesystems, ...)
% Go over to different kernel types, micro, monolith, something between and from there to linux and zircon
