% !TEX root = MasterThesis.tex

\chapter{Modern Operating System Concepts}\label{ch:modern-os-concepts}
%
%https://groups.google.com/forum/#!topic/comp.os.minix/wlhw16QWltI%5B1-25%5D
 % 2: your job is being a professor and researcher: That's one hell of a
% good excuse for some of the brain-damages of minix. I can only hope (and
% assume) that Amoeba doesn't suck like minix does.
% >1. MICROKERNEL VS MONOLITHIC SYSTEM
% True, linux is monolithic, and I agree that microkernels are nicer. With
% a less argumentative subject, I'd probably have agreed with most of what
% you said. From a theoretical (and aesthetical) standpoint linux looses.
% If the GNU kernel had been ready last spring, I'd not have bothered to
% even start my project: the fact is that it wasn't and still isn't. Linux
% wins heavily on points of being available now.
%
This thesis' introduction already picked up the discussion about which operating system architecture is the superior one by refering to the \textit{Tanenbaum-Torvalds debate}\cite{linux-is-obsolete} in 1992.
Besides the discussion is quite interessting from todays view on different operating systems, the forecasts on their future and the actual development, are both, \textsc{Tanenbaum} and \textsc{Torvalds} underpinning their arguments with the origins of their implementations \textit{MINIX} and \textit{Linux} in different problems. 
And roughly spoken are exactly such different problems which needs to be solved with an operating system one reason for their diversity. 
Over the years, they had to fit in solving very different kinds of problems on very different kinds of hardware, which resulted in many different ways of working and architectures. 
As with the debate, it is rather difficult to impossible to find one architectural concept or implementation which is clearly superior to the other ones in every use-case. 
Nevertheless, it is a reasonable question why a majority of the operating system kernels which were developed from scratch in the last few years are based on a microkernel concept.
In 2009, the official statement of the Linux kernel developer \textsc{Richard Gooch} was still that monolithic kernels are superior for performance reasons\cite{why-linux-monolith}.
So the question remains what changed during the last ten years to promote this change, and especially for this work, how this affects on device driver development.
For this reason, this chapter is dedicted to the basic components and functionallities of operating system kernels and how they are implemented in the real world examples \textit{Linux} and \textit{Zircon}.


%%%%%%%%%%%%%%%%%%%%%%%%%%%%%%%%%%%%%%%%%%%%%%%%%%%%%%%%%%%%%%%%%%%%%%%%
% Links
%
% Some notes to Linux design decisions: \url{http://vger.kernel.org/lkml/#s15-3}
% Memory management in Linux -> Files from LFD 430
% 
%
%%%%%%%%%%%%%%%%%%%%%%%%%%%%%%%%%%%%%%%%%%%%%%%%%%%%%%%%%%%%%%%%%%%%%%%%


\section{Operating System Architectures}\label{sec:kernel-arch-concepts}

% TODO OS design goals see Betriebssysteme Glatz 
As already pointed out are architectural decisions for operating systems commonly influenced by the issues they are intended to solve.
By giving priority to some design objectives that are pertinent to the underlying issue, different concepts and architectures are the outcome.
According to \textsc{Glatz}\cite{glatz2015betriebssysteme} are some of them:
\begin{itemize}
    \item Providing a reliable, crash-proof environment.
    \item Providing a portable operating system.
    \item Providing a scalable operating system, e.g.\ in terms of processing cores.
    \item Providing an extensibile operating system, e.g.\ in terms of adding additional functionality to the kernel.
    \item Providing real-time capabilities.
    \item Providing an efficient design in terms of ressources and performance.
    \item Providing a secure environment for user applications.
    \item Providing a maintainable operating system, e.g.\ by the division of policy and mechanism. 
\end{itemize}\ \\
%
How these design principles fit into the known operating system architectures will be considered in the following sections.
But before, the terms \textit{kernel mode} and \textit{user mode} will be explained for a better understanding.

%dual mode is supported by hardware in common -> issue if not
tanenbaum, s 23(54)
achilles, s 11(25)
os concepts 8, s 22(36)

% TODO kernel mode Achilles direkt vor 1.3.1!
% TODO no extra entry for multicore OS. That's nothing special today, defacto standard SMP?
% TODO mischformen
\subsection{Monolithic Architectures}
\subsection{Microkernel Architectures}
% TODO Why are microkernel interesting and nearly each new developed one is a microkernel?
\subsection{Layered Architectures}
% \subsection{Distributed Architectures}
\subsection{Hybrid Architectures}
\subsection{The Linux Kernel's Monolithic Architecture}
% TODO actual Linux architecture, mixed form of monolith, module concept
\subsection{The Zircon Kernel's Microkernel Architecture}
% TODO actual Zircon architecture, overview


\section{System Calls} %TODO label
% TODO normally are system calls a part of the section processes and threads,
% but for the reason zircon is very special there (not fully POSIX compatible),
% it shall be handled with more respect in this work
% (If it will work)? -> it should -> Achilles Betriebssysteme
% Tanenbaum 1.6
% Operating system concepts 8.edition 2.3
% Betriebsysteme Grundlagen und Konzepte 1.3.1
\subsection{POSIX}
\subsection{System Calls in Linux}
\subsection{System Calls in Zircon}
% FIDL, core libs,.. 


\section{Processes and Threads} %TODO label
\subsection{Processes}
\subsection{Threads}
\subsection{Inter Process Communication}
\subsection{Scheduling}
\subsection{Processes and Threads in Linux}
\subsection{Processes and Threads in Zircon}


\section{Memory Management} %TODO label
%    Linux Concept -> LFD430 pic
%TODO find sections
%TODO how is it done in Linux and Zircon
\subsection{Address Spaces}
\subsection{Virtual Memory}
\subsection{Page Replacement/Paging}
\subsection{Memory Management in Linux}
\subsection{Memory Management in Zircon}


\section{I/O} %TODO label
%TODO review sections, how is it done in Linux/Zircon
\subsection{I/O Hardware}
\subsection{Memory Mapped IO}
\subsection{Direct Memory Access}
\subsection{Interrupts}
\subsection{Power Management}

   
\section{Security Concepts} %TODO label
\subsection{Access Control} %Domains, Capabilities
%DAC, MAC, ..
\subsection{Security Concepts in Linux}
\subsection{Security Concepts in Zircon}


\section{Driver Models} %TODO label
