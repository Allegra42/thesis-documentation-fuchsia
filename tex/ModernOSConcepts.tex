% !TEX root = MasterThesis.tex

\chapter{Modern Operating System Concepts}\label{ch:modern-os-concepts}
%
%https://groups.google.com/forum/#!topic/comp.os.minix/wlhw16QWltI%5B1-25%5D
 % 2: your job is being a professor and researcher: That's one hell of a
% good excuse for some of the brain-damages of minix. I can only hope (and
% assume) that Amoeba doesn't suck like minix does.
% >1. MICROKERNEL VS MONOLITHIC SYSTEM
% True, linux is monolithic, and I agree that microkernels are nicer. With
% a less argumentative subject, I'd probably have agreed with most of what
% you said. From a theoretical (and aesthetical) standpoint linux looses.
% If the GNU kernel had been ready last spring, I'd not have bothered to
% even start my project: the fact is that it wasn't and still isn't. Linux
% wins heavily on points of being available now.
%
This thesis' introduction already picked up the discussion about which operating system architecture is the superior one by refering to the \textit{Tanenbaum-Torvalds debate}\cite{linux-is-obsolete} in 1992.
Besides the discussion is quite interessting from todays view on different operating systems, the forecasts on their future and the actual development, are both, \textsc{Tanenbaum} and \textsc{Torvalds} underpinning their arguments with the origins of their implementations \textit{MINIX} and \textit{Linux} in different problems. 
And roughly spoken are exactly such different problems which needs to be solved with an operating system one reason for their diversity. 
Over the years, they had to fit in solving very different kinds of problems on very different kinds of hardware, which resulted in many different ways of working and architectures. 
As with the debate, it is rather difficult to impossible to find one architectural concept or implementation which is clearly superior to the other ones in every use-case. 
Nevertheless, it is a reasonable question why a majority of the operating system kernels which were developed from scratch in the last few years are based on a microkernel concept.
In 2009, the official statement of the Linux kernel developer \textsc{Richard Gooch} was still that monolithic kernels are superior for performance reasons\cite{why-linux-monolith}.
So the question remains what changed during the last ten years to promote this change, and especially for this work, how this affects on device driver development.
For this reason, this chapter is dedicted to the basic components and functionallities of operating system kernels and how they are implemented in the real world examples \textit{Linux} and \textit{Zircon}.


%%%%%%%%%%%%%%%%%%%%%%%%%%%%%%%%%%%%%%%%%%%%%%%%%%%%%%%%%%%%%%%%%%%%%%%%
% Links
%
% Some notes to Linux design decisions: \url{http://vger.kernel.org/lkml/#s15-3}
% Memory management in Linux -> Files from LFD 430
% 
%
%%%%%%%%%%%%%%%%%%%%%%%%%%%%%%%%%%%%%%%%%%%%%%%%%%%%%%%%%%%%%%%%%%%%%%%%


\section{Operating System Architectures}\label{sec:kernel-arch-concepts}

% TODO OS design goals see Betriebssysteme Glatz 
As already pointed out are architectural decisions for operating systems commonly influenced by the issues they are intended to solve.
By giving priority to some design objectives that are pertinent to the underlying issue, different concepts and architectures are the outcome.
According to \textsc{Glatz}\cite{glatz2015betriebssysteme} are some of them:
\begin{itemize}
    \item Providing a reliable, crash-proof environment.
    \item Providing a portable operating system.
    \item Providing a scalable operating system, e.g.\ in terms of processing cores.
    \item Providing an extensibile operating system, e.g.\ in terms of adding additional functionality to the kernel.
    \item Providing real-time capabilities.
    \item Providing an efficient design in terms of ressources and performance.
    \item Providing a secure environment for user applications.
    \item Providing a maintainable operating system, e.g.\ by the division of policy and mechanism. 
\end{itemize}\ \\
%
In addition, operating systems should also pay attention to the common software design issue \textit{mechanism vs.\ policy}.
That means an operating system design should provide a clear distinction between the \textit{mechanism}, that means the capable abilities that can be performed (how is something done) and the \textit{policy}, which controlls how the available capabilities are used (what is done)\cite{lfd430},~\cite{silberschatz2009operating}.
An example for driver development could be controlling the number of processes that can use a device at once.
In this case, the driver should provide the mechansim, the \textit{how} such a limitation could be done, but not \textit{what}, the actual number of allowed processes. 
The idea behind is that requirements may change over some time and such a distinction makes it easy to adjust the \textit{policy} via some parameters without touching the underlying \textit{mechanism}\cite{silberschatz2009operating}.

How these design principles fit into the known operating system architectures will be considered in the following sections.
But before, the terms \textit{kernel mode} and \textit{user mode} will be explained as they are fundamental for this work.

\subsubsection*{Dual-Mode Execution}
Modern general purpose \acp{cpu} provide a ring based, hardware enabled security model which had it's origin in the Intel x86 processor architecture\cite{tanenbaum-modern-operating-systems}.
It is usually made of four different security levels, the rings 0 to 3 which are illustrated in Figure~\ref{pic:x86rings}.
In this model, ring 3 is the least secure level, used for common user applications (even if started with extended privileges (\textit{root} for the UNIX-like world)), while ring 2 is used for libraries shared between user applications and ring 1 is for system calls\cite{glatz2015betriebssysteme}.
System calls provide the transition to ring 0, the one with the topmost security level, which is used for the operating system kernel.
As a crucial part of an operating system, they will be discussed in more detail later in this work.

Directly related to this model is the \textit{dual-mode} execution mode of modern \acp{cpu}.
It is a hardware enabled security concept to provide a distinction between the user applications in ring 3 and the actual operating system kernel in ring 0.
Just the kernel in ring 0, running in the \textit{kernel mode} (or \textit{privileged mode, supervisor mode or system mode}), has direct and privileged access to memory, hardware, timers or interrupts, e.g.\ for performing \ac{io} operatings or memory mappings\cite{lfd430}.
User applications in ring 3, running in the \textit{user mode}, are not allowed to them so directly, they have limited privileges and a limited instruction set.
As named above, they need to use a mechanism called \textit{system calls} to transfer the execution to the \textit{kernel mode} where the priviledged actions are performed.
Lastly, the execution is transfered back to the calling user process and with this, the mode changes back to \textit{user mode}.
Figure~\ref{pic:mode-switches} pictures the operating flow of a system call including the mode switches between \textit{user} and \textit{kernel mode}.

%
%
\begin{figure} [ht]
	\centering
	\includegraphics[scale=0.6]{x86Rings}
	\caption{The Rings of the x86's security concept\cite{glatz2015betriebssysteme}}\label{pic:x86rings}
    %TODO own picture!
\end{figure}
%
\begin{figure} [ht]
	\centering
	\includegraphics[scale=0.6]{mode-switch}
    \caption{A system calls sequence including the mode switches\cite{glatz2015betriebssysteme}}\label{pic:mode-switches}
    %TODO own picture!
\end{figure}

The \ac{cpu}'s operating mode is usually controlled by a specific bit in the \ac{psw}\cite{tanenbaum-modern-operating-systems}.
It influences the state of each \ac{cpu} core itself in a multi-processor system, but not the operating system kernel.
As a result, different \ac{cpu} cores may be in a different execution mode\cite{lfd430}.
With this seperation, any priviledged instruction is forbidden in \textit{user mode} and will not be executed.

Based on the dual mode execution on the \ac{cpu}, different architectural concepts for operating systems evolved.
They differ e.g.\ in the share of the operating system respectively the operating system's kernel actually running within the \ac{cpu}'s \textit{kernel mode}. 
Thus, they have an influence on the whole system, including device driver development but also on performance and security issues.

With this basic knowledge about the \ac{cpu}'s operating modes, the next section researches a selection of different operating system architectures.
Special attention should be paid to the most common ones, the \textit{monolithic} and the \textit{microkernel} architectures and their implementation in Linux and Zircon.
On the contrary, this work will not take a particular look on special purpose operating system architectures such as ones for loosely coupled multi-processor systems known from processing clusters.
Today, even the majority of general purpose computing systems are driven by more than one \ac{cpu} and most of common modern operating systems are designed to provide support for the defacto standard for tightly coupled systems, \ac{smp}.


\subsection{Monolithic Architectures}\label{sec:monolithic-archs}
Some sources, such as \textsc{Glatz}\cite{glatz2015betriebssysteme} or \textsc{Silberschatz}\cite{silberschatz2009operating}, suggest monolithic operating systems do not have a well-defined structure at all. 
As they are indeed most commonly grown structures, started in a completely different scope (MS-DOS, the original UNIX), it is not an incorrect claim.
But it does not neccessarily have to be the case.
Above all, monolithic operating system (kernel) architectures have in common that they form one single binary program which is running entirely in kernel mode.
User programms, running in user mode, interact with the kernel only through a well defined set of \textit{system calls}\cite{lfd430}. 
Within the kernel itself, all parts are free to use and access each other but also the hardware, without any limitation, e.g.\ regarging the access of kernel functionalities of another component or hardware acces. 
That means a function or procedure initial developed for scheduling processes could be used in a completely different context if its functionality is useful to solve another issue.
In fact, there is no information hiding between kernel functions or procedures.
Any function in this kernel context has full access to the hardware, such as \ac{io} devices, timers, interrupts and even to the memory. 
There is no memory protection or validation between different components of a monolithic kernel. 
Of course, this leads to some serious disadvantages in this architecture, for example could a crash in one single function or procedure crash the entire kernel or the resulting system may become difficult to understand and maintain\cite{tanenbaum-modern-operating-systems},~\cite{silberschatz2009operating}.
The missing memory protecting within the kernel could also be a source for crashes or attacks.
But in contrast, this design also enables a very efficient kernel design without any unneeded communication overhead or hardware inefficiencies\cite{lfd430}.

An extension of the monolithic architecture are the so-called \textbf{modular operating systems}.
They provide additional, defined interfaces for (in common) dynamically loadable and unloadable extentions, e.g.\ for device drivers or filesystems. 
Sometimes, such extensions or modules are just allowed to use a limited function set of the operating system, but they are still running as a part of the kernel in kernel mode\cite{lfd430},~\cite{tanenbaum-modern-operating-systems}. 
Just like ordinary kernel functions or procedures, (malicious) programming errors in extentions may lead to a kernel crash or manipulate or damage other components.
Contrary, the modular concept provides some advantages over regular monolitic kernels.
It allows to slim down the actual kernel by providing the chance to reload only the actually needed functionality dynamically and e.g.\ security patches within such an extension is possible without restarting the entire system\cite{brause2017betriebssysteme}.
As the extensions become an part of the operating system running in kernel mode, no additional communication effort between the actual kernel and the modules is required.
Thus, concept of modules is quite popular for basically monolithic operating systems like \textit{Linux} or \textit{Solaris}\cite{brause2017betriebssysteme},~\cite{silberschatz2009operating}.


\subsection{Microkernel Architectures}\label{sec:microkernel-archs}
The microkernel architecture focuses on very opposite design goals compared to the monolithic architecture. 
One is to cope the complexity, rather poor maintainablity and susceptibility to errors by a massively modular approach. 
To archive this, the core idea behind microkernels is to provide only a very small kernel running in kernel mode which only provides the core functionalities while all the other important functions of an operating system are running in user mode.
Thereby, the microkernel architecture is excellently suited to implement a proper division of mechanism and policy.
The kernel provides just the most basic mechanisms needed for an operating system, while the userspace modules implement the policy.
This decoupling makes it easier to change the policy in userspace for altering requirements without touching the actual kernel\cite{tanenbaum-modern-operating-systems}.

What is part of this core functionality differs between miscellaneous sources, but all considered ones are in agreement that a simple mechanism for process scheduling is as well a core functionality as providing an \ac{ipc} mechanism\cite{lfd430},~\cite{silberschatz2009operating},~\cite{glatz2015betriebssysteme}.  
In contrast, the sources disagree as to whether memory management and virtualisation, device drivers or synchronization facilities are a part of the actual kernel.
The \textit{Mach} microkernel, which formed the first generation of microkernels in 1985, named process and thread administration, an extensible and secure \ac{ipc} mechanism, virtual memory management and scheduling as its core tasks, while everything else needed has to run in usermode\cite{rashidMach}.
Functional enhancements of the system do not require changes to the kernel itself, too.
This concerns, depending on the exact relization, device drivers, memory management, system call handlers and even more system components\cite{lfd430},~\cite{silberschatz2009operating}.
In academic microkernel approaches, all components in user mode run within an own userspace process as small, well-defined modules, while the communication is done through copious message passing via the actual kernel\cite{tanenbaum-modern-operating-systems},~\cite{lfd430}.
Since the restrictions by the \ac{cpu}'s dual mode still apply for microkernels is it not allowed to device drivers running in user mode to have direct physical access to \ac{io} ports as a consequence.
A device driver has to invoke the actual kernel to perform the needed action substitional.
But thus, the kernel is able to check the action and whether the driver is authorized to executed them.
As a result, the microkernel design is more reliable and secure as such a division enables the kernel to intercept erroneous actions such as accidental memory writes to important regions\cite{tanenbaum-modern-operating-systems}.
Equally a crash in a userspace system component like a driver is not able to crash the entire kernel in such an approach. 
And as an additional advantage facilitate the microkernel architecture porting the operating system kernel to another target architecture as the most hardware dependencies are part of the small kernel\cite{silberschatz2009operating},~\cite{lfd430}.

With all the named advantages microkernels offer, the question remains why microkernels are only spread in real-time, avionics or military but not for desktop operating systems.
One reason is that all these advantages are bought at the high price of microkernel message costs.
For the named application areas, especially the reliability that comes with the microkernel architecture is more desirable than the performance costs of the lot more context switches in comparison to monolithic architectures\cite{tanenbaum-modern-operating-systems}.
Since a lot of the operating system's functionality has been moved to the userspace, microkernel architectures need to perform noticable more context switches to invoke the actual kernel for privileged actions. 
The performance losses are not only caused by the large amount of context switches themselves, but also by the fact that modern \acp{cpu}, particularly the caches are not designed for them. 
Every context switch causes cache misses which trigger that the required data has to be loaded from the slower main memory and cached. 
The data of the previous context (e.g.\ the user mode context) will be displaced from the cache and the \ac{cpu} is largely blocked in the meantime.
By rapidly switching back to the previous context, as is usual for e.g.\ a short kernel invokation to perform an \ac{io} operation on microkernel architectures, the cache is no longer suitable for the new context and has to be replaced\cite{lfd430}.

First of all, the \textit{L4} kernel, a second generation microkernel was able to get close to the performance of a monolithic kernel as \textit{Linux} it is\cite{Hrtig1997}.   
Nevertheless are pure microkernels mainly used for systems with high reliability requirements but unusual for desktop application. 
Some industry examples are \textit{Integrity}, \textit{QNX} and \textit{seL4}, a mathematically verified version of the \textit{L4} kernel\cite{tanenbaum-modern-operating-systems}. 

%TODO ? reincarnation server : check if all modules are up running and work correctly -> if not it replaces them without user interaction \cite{tanenbaum}
%TODO picture \cite{microkernels}


\subsection{Layered Architectures}


% \subsection{Distributed Architectures}


\subsection{Hybrid Architectures}


\subsection{The Linux Kernel's Monolithic Architecture}
% TODO mischformen
% TODO actual Linux architecture, mixed form of monolith, module concept


\subsection{The Zircon Kernel's Microkernel Architecture}

% TODO actual Zircon architecture, overview


\section{System Calls}\label{sec:system-calls}
% TODO normally are system calls a part of the section processes and threads,
% but for the reason zircon is very special there (not fully POSIX compatible),
% it shall be handled with more respect in this work
% (If it will work)? -> it should -> Achilles Betriebssysteme
% Tanenbaum 1.6
% Operating system concepts 8.edition 2.3
% Betriebsysteme Grundlagen und Konzepte 1.3.1
\subsection{POSIX}
\subsection{System Calls in Linux}
\subsection{System Calls in Zircon}
% FIDL, core libs,.. 


\section{Processes and Threads} %TODO label
\subsection{Processes}
\subsection{Threads}
\subsection{Inter Process Communication}
\subsection{Scheduling}
\subsection{Processes and Threads in Linux}
\subsection{Processes and Threads in Zircon}


\section{Memory Management} %TODO label
%    Linux Concept -> LFD430 pic
%TODO find sections
%TODO how is it done in Linux and Zircon
\subsection{Address Spaces}
\subsection{Virtual Memory}
\subsection{Page Replacement/Paging}
\subsection{Memory Management in Linux}
\subsection{Memory Management in Zircon}


\section{I/O} %TODO label
%TODO review sections, how is it done in Linux/Zircon
\subsection{I/O Hardware}
\subsection{Memory Mapped IO}
\subsection{Direct Memory Access}
\subsection{Interrupts}
\subsection{Power Management}

   
\section{Security Concepts} %TODO label
\subsection{Access Control} %Domains, Capabilities
%DAC, MAC, ..
\subsection{Security Concepts in Linux}
\subsection{Security Concepts in Zircon}


\section{Driver Models} %TODO label
