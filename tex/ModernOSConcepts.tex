% !TEX root = MasterThesis.tex

%General Operating System Concepts, not specific to Linux or Zircon at all
\chapter{Modern Operating System Concepts} \label{ch:modern-os-concepts}

% TODO general concepts, say that specific ones follow in the next chapters

\section{Monolithic and Microkernel in Comparison} \label{sec:monolith-vs-microkernel}

%https://groups.google.com/forum/#!topic/comp.os.minix/wlhw16QWltI%5B1-25%5D

 2: your job is being a professor and researcher: That's one hell of a
good excuse for some of the brain-damages of minix. I can only hope (and
assume) that Amoeba doesn't suck like minix does.

>1. MICROKERNEL VS MONOLITHIC SYSTEM
True, linux is monolithic, and I agree that microkernels are nicer. With
a less argumentative subject, I'd probably have agreed with most of what
you said. From a theoretical (and aesthetical) standpoint linux looses.
If the GNU kernel had been ready last spring, I'd not have bothered to
even start my project: the fact is that it wasn't and still isn't. Linux
wins heavily on points of being available now.



% TODO general comparison between both concepts.
% Why are microkernel interesting and nearly each new developed one is a microkernel?

\section{Processes and Threads} %TODO label

\section{Memory Management} %TODO label
    Linux Concept -> LFD430 pic

\section{I/O} %TODO label

\section{System Calls} %TODO label
    POSIX
