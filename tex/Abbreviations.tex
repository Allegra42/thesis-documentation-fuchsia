% !TEX root = MasterThesis.tex

\chapter*{List of Abbreviations}

\begin{acronym} [POSIX]
% a
\acro{api} [API] {Application Programming Interface}
    \acro{apsr} [APSR] {Application Program Status Register}
% b

% c
\acro{cpsr} [CPSR] {Current Program Status Register}
\acro{cpu} [CPU] {Central Processing Unit}

% d

% e
\acro{elf} [ELF] {Executable and Linking Format}

% f
\acro{fifo} [FIFO] {First In First Out}
\acro{futex} [futex] {fast user-space mutex}

% g
\acro{gcc} [GCC] {GNU Compiler Collection}

% h

% i
\acro{idl} [IDL] {Interface Definition Langugage}
\acro{io} [I/O] {Input/Output}
\acro{ipc}[IPC] {Inter-Process Communication}

% j

% k

% l
\acro{lwp} [LWP] {Light-Weight Process}

% m

% n

% o

% p
\acro{pc} [PC] {Program Counter}
\acro{pcb} [PCB] {Process Controll Block}
\acro{pid} [PID] {Process Identifier}
\acro{posix} [POSIX] {Portable Operating System Interface for UniX}
\acro{psw} [PSW] {Program Status Word}

% q

% r
    \acro{rpc} [RPC] {Remote Procedure Call}

% s
\acro{smp} [SMP] {Symmetric Multiprocessing}

% t
\acro{tas} [TAS] {Test and Set}
\acro{tcb} [TCB] {Thread Controll Block}
\acro{tsl} [TSL] {Test and Set Lock}

% u

% v
\acro{vdso} [vDSO] {virtual Dynamic Shared Object}
\acro{vmar} [VMAR] {Virtual Memory Address Region}

% w

% x
\acro{xchg} [XCHG] {Exchange}

% y

% z

\end{acronym} 
